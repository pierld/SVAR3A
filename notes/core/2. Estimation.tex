% ============================================
% ============================================
\subsection{Classification from \textit{Shapiro (2022)}}
\subsubsection{Baseline classification}

\quad With our price-quantity pairs matched for 75 sectors, we run the following monthly VAR for each of the sectors denoted by index ($i$) :
\begin{align*}
    \begin{bmatrix} \Delta q_{i,t} \\ \Delta p_{i,t} \end{bmatrix} = \mathlarger{\mathlarger{\sum}}_{j=1}^{p}A_{j}.\begin{bmatrix} \Delta q_{i,t-j} \\ \Delta p_{i,t-j} \end{bmatrix} + \begin{bmatrix} \nu_{i,t}^{q} \\ \nu_{i,t}^{p} \end{bmatrix}\;
\end{align*}

Where $\Delta q_{i,t}$ and $\Delta p_{i,t}$ are the demeaned first-difference of the log-transformed quantity and price indices ($x_{i,t} = \textrm{log}(X_{i,t})$).
This assumes $\Delta q_{i,t}$ and $\Delta p_{i,t}$ are second-order stationary, which seems to be the case. 
\bigbreak
In our implementation of the methodology, we let the order $p$ of the VAR be automatically selected via minimisation of a statistical criterion (AIC or BIC) for each sector instead of imposing $p=12$ as done by Shapiro.
Even though he explains the classification seems to be robust to the lag order selection we would rather let it be sector-specific than adding another constraint on the methodology.
\bigbreak
Note that we said 75 sectors had matched proxy quantity series, but we cannot run 75 VAR. Indeed, even though we managed to match the series we still had some data missing in the series themselves.
Thus, for each zone in our study (France, Germany, Spain and EU27) we flag sectors that have data problems for either the price or quantity series. A sector being flagged mainly stems from the latest data available being earlier than 2023 (some stop in 2022 or 2020 and some even in 2015).
Overall we run regressions on 52 (EU27) to 72 (France) sectors.
\bigbreak
The estimated reduced-form residuals $(\nu_{t}^{q},\nu_{t}^{p})$ (equivalent to $(\nu_{t}^{s},\nu_{t}^{d})$ in the previous section) are then used to label each month $(t)$ for each sector $(i)$ using the following rule
\begin{align}
    \ind_{i,sup(+)}(t) &= \begin{cases} 1 \quad \textrm{if} \; \nu_{i,t}^{q}>0,\nu_{i,t}^{p}<0 \\ 0 \quad \textrm{otherwise} \end{cases} \nonumber\\
    \ind_{i,sup(-)}(t) &= \begin{cases} 1 \quad \textrm{if} \; \nu_{i,t}^{q}<0,\nu_{i,t}^{p}>0 \\ 0 \quad \textrm{otherwise} \end{cases} \nonumber\\
    \ind_{i,dem(+)}(t) &= \begin{cases} 1 \quad \textrm{if} \; \nu_{i,t}^{q}>0,\nu_{i,t}^{p}>0 \\ 0 \quad \textrm{otherwise} \end{cases} \nonumber\\
    \ind_{i,dem(-)}(t) &= \begin{cases} 1 \quad \textrm{if} \; \nu_{i,t}^{q}<0,\nu_{i,t}^{p}<0 \\ 0 \quad \textrm{otherwise} \end{cases} \nonumber\\
    \Rightarrow \ind_{i,sup}(t) &= \ind_{i,sup(+)}(t) + \ind_{i,sup(-)}(t) \label{shapiro_sup}\\ 
    \Rightarrow \ind_{i,dem}(t) &= \ind_{i,dem(+)}(t) + \ind_{i,dem(-)}(t) \label{shapiro_dem}\\ \nonumber
\end{align}

\textit{From now on, at date $t$ experiencing a demand shock or being demand-driven or being labeled demand-driven all mean that $ \ind_{i,dem}(t)=1$ (vice versa with supply).}
\bigbreak
It follows that each month, overall month-over-month inflation $\pi_{t,t-1}$ can be devided into supply and demand driven components. 
As the overall CPI is a weighted average of its sectorial inflation $\pi_{i,t,t-1}$ we denote $\omega_{i,y}$ the weight of sector $(i)$ at year $y$ in the CPI.
In the following equation $y_t$ means the year of month $t$.
\begin{align*}
    \pi_{t,t-1} &= \mathlarger{\mathlarger{\sum}}_{i} \ind_{i,sup}(t) \omega_{i,y_t} \pi_{i,t,t-1} + \mathlarger{\mathlarger{\sum}}_{i} \ind_{i,dem}(t) \omega_{i,y_t} \pi_{i,t,t-1} \\
    \pi_{t,t-1} &= \pi_{t,t-1}^{sup} + \pi_{t,t-1}^{dem}
\end{align*}

We present results for year-over-year inflation, which is simply given by:
\begin{align*}
    \begin{cases}
        \pi_{t,t-12}^{sup} &= \mathlarger{\mathlarger{\sum}}_{k=0}^{11} \pi_{t-k,t-k-1}^{sup} \\
        \pi_{t,t-12}^{dem} &= \mathlarger{\mathlarger{\sum}}_{k=0}^{11} \pi_{t-k,t-k-1}^{dem}
    \end{cases}
\end{align*}

\subsubsection{Robustness check : alternative classifications}

\quad Shapiro introduced several alternative classification methods which we replicated to assess whether the methodology was robust enough.
In our version, as the lag order for each VAR is automatically selected (and if we believe the selected lag is correct) there should not be specification error.
\bigbreak
We implement the the two alternative methods that are the smoothed-errors classification (to offset possible measurement errors) and the parametric approach. 
In the latter, each month a sector is no longer deterministically classified as experiencing a supply or a demand shock but there rather is a probability $p_t^K$ that it is experiencing a shock of type $K \in$\{supply,demand\}.
\bigbreak
The smoothed-errors classification amounts to considering $\sum_{k=0}^{j} \nu_{i,k}^{q}$ and $\sum_{k=0}^{j} \nu_{i,k}^{p}$ instead of simply $\nu_{i,t}^{q}$ and $\nu_{i,t}^{p}$ in the aforementioned classification. 
The parameter $j$ is an hyperparamter, set to be the same for each sector. In his paper Shapiro tries different values (ranging from 1 to 3) meaning for each month $t$ it is the sign of the (average) $j$ previous months residuals that matter.


% ============================================
% ============================================
\subsection{Classification from \textit{Sheremirov (2022)}}

\quad The algorithm proposed by Sheremirov is as follows, for each sector $i$ and month $t$ :
\begin{itemize}
    \item[1.] Compute year-over-year inflation and (real) consumption/demand growth : $\pi_{i,t},c_{i,t}$
    \item[2.]\label{sherem2} Classify the sector $i$ as experiencing a demand shock at month $t$ if both inflation and consumption growth are over their 2001-2019 respective average $\tilde{\pi_i},\tilde{c_i}$ : 
    \begin{align*}
        \ind_{i,dem}(t) &= \begin{cases} 1 \quad \textrm{if} \; (\pi_{i,t}-\tilde{\pi_i}).(c_{i,t}-\tilde{c_i})>0 \\ 0 \quad \textrm{otherwise} \end{cases} \\ 
        \ind_{i,sup}(t) &= 1 - \ind_{i,dem}(t)
    \end{align*}
    \item[3.] Classify the shock at month $t$ as:
    \begin{align*}
        \textrm{Persistent demand} \; \ind_{i,dem}^{pers}(t) &= \begin{cases} 1 \quad \textrm{if} \; \mathlarger{\mathlarger{\sum}}_{k=0}^{11} \ind_{i,dem}(t-k) \geq 11 \\ 0 \quad \textrm{otherwise} \end{cases} \\ 
        \textrm{Persistent supply} \; \ind_{i,sup}^{pers}(t) &= \begin{cases} 1 \quad \textrm{if} \; \mathlarger{\mathlarger{\sum}}_{k=0}^{11} \ind_{i,sup}(t-k) \geq 11 \\ 0 \quad \textrm{otherwise} \end{cases} \\ 
    \end{align*}
    \item[4.] If neither condition is verified, month $t$ is classified as transitory demand (if $\ind_{i,dem}(t)=1$) or transitory supply.
\end{itemize}
\quad The Sheremirov methodology was implemented to assess to what extent classifications derived from Shapiro's methods would be consistent with alternative methods, especially one not based on a VAR estimation. 
However the distinction it provides between transitory and persistent components makes it really interesting for policy-makers even though the classification is solely backward-looking.
\bigbreak

\subsection{Proposed classification rule}

\quad On this basis we suggest tweaking the classification outlined by Sheremirov, challenging the set number of months (11) for the persistent classification and the pure backward-looking perspective. 
\bigbreak
The latter can be a problem as the classification does not rule out persistent demand (or supply) vanishing. 
Consider a sector $i$ and assume month $t-1$ is not labeled as persistent demand because months $t-10$ and $t-12$ are not labeled as demand-driven and let month $t$ be demand-driven. 
According to Sheremirov's rule, month $t$ is labeled as persistent demand as we have 11 months classified as demand-driven (only month $t-10$ is not).
However, if month $t+1$ is not demand-driven, persistent demand vanishes after only one month because at least 2 months are not demand-driven conflicting the criterion.
Which in our opinion is kind of an awkward situation as the whole purpose of the persitent component is to be persistent and not to instantaneously disappear. 
Our methodology solves this, as month $t$ would have never been labeled as persistent demand in the first place due to the forward-looking setting we incorporate to the rule.
\bigbreak
On another note, as the first step of the classification below relies on the simple dummy demand-supply classification, we can use both results from the labeling of Shapiro or (step.2 of) Sheremirov.
Our aim is to leverage the initial supply-demand classification, decomposing each as transitory-persistent (as Sheremirov did) with both backward and forward-looking perspectives.
We add sort of a feedback rule -an ambiguous category- such that the dates for which the forward-looking perspective is not possible (due to unobserved data) are not miss-classified.
A month that had been categorized as ambiguous can thus be re-classifed once "enough" data have been added.
\bigbreak
We propose the following algorithm:
\begin{itemize}
    \item[1.] Classify month $t$ for sector $i$ as experiencing a demand or supply shock: either through Shapiro's rule (\ref{shapiro_sup},\ref{shapiro_dem}) or step.2 of Sheremirov's algorithm (\ref{sherem2}).
    \item[2.] We introduce the following modified rule, where $K$ can be debated but we set the baseline as $K=3$:
    \begin{align*}
    %    \textrm{Persistent demand} \; \ind_{i,dem}^{pers}(t) &= \begin{cases} 1 \quad \textrm{if} \; \ind_{i,dem}(t-k)=\ind_{i,dem}(t+k) = 1 \; \textrm{for}\; k \in[0,K] \\ 0 \quad \textrm{otherwise} \end{cases} \\ 
    %    \textrm{Persistent supply} \; \ind_{i,sup}^{pers}(t) &= \begin{cases} 1 \quad \textrm{if} \; \ind_{i,sup}(t-k)=\ind_{i,sup}(t+k) = 1 \; \textrm{for}\; k \in[0,K] \\ 0 \quad \textrm{otherwise} \end{cases} \\ 
        \textrm{Persistent demand} \; \ind_{i,dem}^{pers}(t) &= \begin{cases} 1 \quad \textrm{if} \; \; \ind_{i,dem}(t) + \mathlarger{\mathlarger{\sum}}_{k=1}^{K} \{ \ind_{i,dem}(t-k)+\ind_{i,dem}(t+k) \} \geq 2K \\ 0 \quad \textrm{otherwise} \end{cases} \\ 
        \textrm{Persistent supply} \; \ind_{i,sup}^{pers}(t) &= \begin{cases} 1 \quad \textrm{if} \; \; \ind_{i,sup}(t) + \mathlarger{\mathlarger{\sum}}_{k=1}^{K} \{ \ind_{i,sup}(t-k)+\ind_{i,sup}(t+k) \} \geq 2K \\ 0 \quad \textrm{otherwise} \end{cases} \\ 
        \textrm{\textit{Ambiguous} demand} \; \ind_{i,dem}^{abg}(t) &= \begin{cases} 1 \quad \textrm{if} \; t\in [T-K+1,T] \; \textrm{and :}\\ \ind_{i,dem}(t) + \mathlarger{\mathlarger{\sum}}_{k=1}^{K} \ind_{i,dem}(t-k) + \ind_{t\neq T}\mathlarger{\mathlarger{\sum}}_{k=1}^{T-t}\ind_{i,dem}(t+k) \geq K+(T-t) \\ \textrm{\textit{i.e almost satisfies pers. but some date '} $t+k$ \textit{' is unobserved!}} \\ 0 \quad \textrm{otherwise} \end{cases} \\ 
        \textrm{\textit{Ambiguous} supply} \; \ind_{i,dem}^{abg}(t) &= \begin{cases} 1 \quad \textrm{if} \; t\in [T-K+1,T] \; \textrm{and :}\\ \ind_{i,sup}(t) + \mathlarger{\mathlarger{\sum}}_{k=1}^{K} \ind_{i,sup}(t-k) + \ind_{t\neq T}\mathlarger{\mathlarger{\sum}}_{k=1}^{T-t}\ind_{i,sup}(t+k) \geq K+(T-t) \\ \textrm{\textit{i.e almost satisfies pers. but some date '} $t+k$ \textit{' is unobserved!}} \\ 0 \quad \textrm{otherwise} \end{cases} \\ 
    %    \textrm{\textit{Ambiguous} demand} \; \ind_{i,dem}^{abg}(t) &= \begin{cases} 1 \quad \textrm{if} \; t\in [T-K+1,T] \; \textrm{and both}\\ \quad \; \; \rightarrow \ind_{i,dem}(t-k)= 1 \; \textrm{for}\; k \in[0,K] \\ \quad \; \; \rightarrow \ind_{i,dem}(t+k)= 1 \; \textrm{for}\; k \in[T-t,T] \\ \textrm{\textit{i.e almost satisfies pers. but some} $t+k$ \textit{is unobserved!}} \\ 0 \quad \textrm{otherwise} \end{cases} \\ 
    %    \textrm{\textit{Ambiguous} supply} \; \ind_{i,sup}^{abg}(t) &= \begin{cases} 1 \quad \textrm{if} \; t\in [T-K+1,T] \; \textrm{and both}\\ \quad \; \; \rightarrow \ind_{i,sup}(t-k)= 1 \; \textrm{for}\; k \in[0,K] \\ \quad \; \; \rightarrow \ind_{i,sup}(t+k)= 1 \; \textrm{for}\; k \in[T-t,T] \\ \textrm{\textit{i.e almost satisfies pers. but some} $t+k$ \textit{is unobserved!}} \\ 0 \quad \textrm{otherwise} \end{cases} \\ 
        %\textrm{\textit{Ambiguous} demand} \; \ind_{i,dem}^{abg}(t) &= \begin{cases} 1 \quad \textrm{if} \; \ind_{i,dem}(t-k)=\ind_{i,dem}(t+k^{'}) = 1 \; \textrm{for}\; k \in[1,K] \; \textrm{and} \\ \quad \; \; k^{'}\in[1,K_{<T}] \;\textrm{where}\; t+K_{t}=T \;, 0\le K_{t}<K \\ \quad \; \; \textrm{i.e} \; t\in [T-K+1,T]\\ 0 \quad \textrm{otherwise} \end{cases} \\ 
    \end{align*}
    \item[3.] If neither condition is verified, month $t$ is simply classified as transitory demand (if $\ind_{i,dem}(t)=1$) or transitory supply.
\end{itemize}

The persistent demand(supply) lebeling simply asserts that for a given month $t$, in the window of the three previous and next months, at least $2K$ have been classified as experiencing a demand(supply) shock.
For instance when $K=3$ this checks that for Septembre 2019, between June and Decembre 2019 (7-month window), at least 6 have been labeled as demand-driven (for persistent demand and vice versa).
\bigbreak
The \textit{ambiguous} (ambiguous in the sense of undetermined between transitory and persistent) labeling is pretty similar to the persistent but is only applied to the last $K$ observations. 
It aims at labeling the months for which we cannot apply the full rule because the forward-looking part simply sums over unobserved data.
For instance, when $K=3$, only the last three observations ($t\in[T-2,T]=[T-K+1,T]$) can be categorized as ambiguous.
When assessing the label for $t=T-2$ the condition for ambiguous demand checks if in the range of: the previous 3 months + current month + next $T-(T-2)=2$ available months, at least $3+(T-(T-2))=5$ have been labeled as experiencing a demand shock.
\bigbreak
The labeling is retroactive as once data for the date $t=T+1$ is observed the classfication of month $T-2$ -had it been classified as ambiguous- is potentially changed if it either satisfies the "complete" persistent supply or demand criterion. 
Otherwise it is labeled as transitory.
\bigbreak
\textit{NB1: with the sample denoted }$[1,T]$\textit{ we start the classification at date }$t=1+K$

\textit{NB2: the classification suggested above does not work for the parametric approach, as the latter simply does not have dummy variables for supply-demand classfication but rather probabilities, which impedes the counting rule put in place.}


\clearpage